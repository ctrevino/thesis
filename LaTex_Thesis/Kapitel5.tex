% !TeX encoding = ISO-8859-1
%%%%%%%%%%%%%%%%%%%%%%%%%%%%%%%%%%%%%%%%%%%%%%%%%%%%%%%%%%%%
\chapter{Conclusions and Outlook}
\label{ch:concl}
\section{Outlook}
\label{Outlook}
With the advances in autonomous driving 

In this thesis an approach to understand traffic scenes using deep learning methods

\section{Future Work}
\label{Future Work}
The results of this thesis could be expanded in the near future in several ways. Development in Graphic Processor Units, (GPU). Computer Vision challenges (i.e. Pascal VOC, Imagenet, et cera.) keep on showing outperforming results every year. New techniques and methods yield more accurate predictions in vision tasks. GPUs manufacturers are not only developing more powerful hardware, but also improving (CUDNN) These perks will allow to train bigger, more complex models in a faster and more efficient way.  To have a real time semantic segmentation with an good understanding of the surroundings... Information from Pedestrians, Autos, Biciclyst and more important classes could be used for security. Traffic signs, and (pinturas en el asfalto) could be interpreted and used to simplify autonomous driving. Some works (cite paper) already show similar approaches with acceptable resutls. Nevertheless as time passes, technologies are developed and more informationd and more capable devices   
\section{Results and Conclusions}
\label{Results and Conclusions}
The work in this thesis shows that with .
I would like to remark that some already available models have a better performance, but due hardware limitations it was not feasible to train certain models. Some image preprocessing was also mandatory to cope with traffic scene understanding. All the images belonging to the CamVid dataset were downsampled. This probably means that by having input images with greater quality, the learning process inside the CNN would perform better. The following tables show the obtained results on the CamVid dataset by using independent segmentations obtained from a CNN and temporary consistent segmentations. 
We have moved from the HOG and SIFT era to the convolutional networks features era. 