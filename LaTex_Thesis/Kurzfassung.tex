% !TeX encoding = ISO-8859-1
\chapter*{\iftoggle{lang_eng}{Abstract}{Kurzfassung}}
Modern driver assistance systems aim to automate the task of driving a vehi-cle in increasingly complex situations. In order to handle these situations, a profound understanding of the scene surrounding the vehicle is essential. The rich level of detail found in camera images makes them a natural choice to approach this problem. In computer vision, scene labeling is known as la-beling each pixel in an image with the category of the object or region it belongs to. Existing methods rely on learning databased models from recorded and hand-labeled datasets using machine learning techniques. One method are so called convolutional neural network (CNN) models, which are variants of regular neural networks consisting of a number of convolutional and subsam-pling layers optionally followed by fully connected layers. CNN models de-pict the state of the art in several computer vision problems. However, they typically require large datasets and long computation times during the training phase.
One way to overcome this drawback is to use pre-trained, readily para-metrized models as an initialization and adapt (fine-tune) them to solve a specific problem. For this, the thesis task is to analyze, how existing pre-trained CNN models can be adapted towards the problem of traffic scene la-beling. Therefor the method of fully convolutional networks is to be exam-ined, which does not require complex pre- or post-processing steps and al-lows for adapting CNN models trained for whole-image classification towards scene labeling tasks. The task includes preparing an appropriate dataset for traffic scenes using typical category labels including road, car and pedes-trian.
